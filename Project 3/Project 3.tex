\documentclass[12pt]{article}

\usepackage{ amssymb, amsmath, graphicx }

\addtolength{\textheight}{2.2in}
\addtolength{\topmargin}{-1.2in}
\addtolength{\textwidth}{1.661in}
\addtolength{\evensidemargin}{-0.8in}
\addtolength{\oddsidemargin}{-0.8in}
\setlength{\parskip}{0.1in}
\setlength{\parindent}{24pt}

\raggedbottom

\newcommand{\given}{\, | \,}

\begin{document}

\begin{flushleft}

Prof.~David Draper \\
Department of Statistics \\
University of California, Santa Cruz

\end{flushleft}

\begin{center}

\textbf{\large STAT 131: Take-Home Test 3, part 1 (required) \textit{[270 total points]}} \\ Due date: upload to \texttt{canvas.ucsc.edu} by \textbf{11.59pm Sun 14 Jun 2020} 

\end{center}

\Large Name: Christy Yuen \\
\normalsize Note: Attended Professor Draper's Office Hours and got lots of help. 

1.~\textit{[130 total points]} (biology) \textit{Limnology} is the study of inland waters (both saline and fresh), including their biological, chemical and hydrological properties. One common outcome variable in studies in this branch of biology is pH, because the acidity of a lake can be an important factor in
determining the abundance of fish and other wildlife living in and near
it. According to the web site
\texttt{www.lenntech.com/aquatic/acids-alkalis.htm}, 

\begin{quote}

Unpolluted deposition (or rain), in balance with atmospheric carbon
dioxide, has a pH of 5.6. Almost everywhere in the world the pH of rain is
lower than this. The main pollutants responsible for acid deposition (or
acid rain) are sulfur dioxide (SO$_2$) and nitrogen oxides (NO$_x$). Acid
deposition influences mainly the pH of freshwater.~...~Most freshwater
lakes, streams, and ponds have a natural pH in the range of 6 to 8. Acid
deposition has many harmful ecological effects when the pH of most aquatic
systems falls below 6 and especially below 5. Here are some effects of
increased acidity on aquatic systems:

\begin{itemize}

\item

As the pH approaches 5, non-desirable species of plankton and mosses may
begin to invade, and populations of fish such as small-mouth bass
disappear.

\item

Below a pH of 5, fish populations begin to disappear, the bottom is
covered with undecayed material, and mosses may dominate near-shore areas.

\item

Below a pH of 4.5, the water is essentially devoid of fish.

\end{itemize}

\end{quote}

You're a limnologist out in the field studying a lake --- sufficiently
remote that you had to backpack in to get to it --- and this lake looks
like it may already have been damaged by acid rain. The only pH
measurement kit you could bring with you in your backpack is rather crude:
it's known to give unbiased pH measurements that fluctuate around the true
value with an SD of 0.15 and an approximately Normal distribution for its
measurement errors. You'll be surveying enough lakes on this trip that you
can't bring water samples back with you; you need to estimate their pH
values in the field.  

You're wondering if the pH of the lake you're now standing in front of is
below 5; let's agree to call any such lake \textit{threatened}. You decide to take one or more pH measurements to reduce your uncertainty about the lake's status. 

This problem is about \textit{measurement error}, so I need to introduce some notation and concepts.
Before you've measured anything, let $Y_i$ be a random variable capturing the uncertainty in your prediction of observation $i$, as $i$ runs from 1 to $n$. In words, the standard measurement error model encourages you to additively decompose $Y_i$ into the sum of (the true quantity being measured) plus (systematic error, also known as \textit{bias}) + (random error):
\begin{equation} \label{ph-1}
( \text{observation} )_i = ( \text{truth} ) + ( \text{bias} ) + ( \text{random error} )_i \, .
\end{equation}
This model requires an act of imagination to formulate, because the only thing we get to observe (the number on the left side of the equation) is broken into the sum of three things we can't observe; you may therefore wonder at its usefulness, but (as we'll see) it's actually quite helpful.

Let $\theta$ stand for the true value of the thing being measured (in this problem, $\theta$ is the true pH of the lake); let $b$ stand for the bias in the measurement process; and let the $e_i$ be the random measurement errors. Then symbolically equation (\ref{ph-1}) looks like
\begin{eqnarray} \label{ph-2}
Y_1 & = & \theta + b + e_1 \nonumber \\
\vdots & & \vdots \hspace*{0.25in} \vdots \hspace*{0.25in} \vdots \nonumber \\
Y_n & = & \theta + b + e_n \, .
\end{eqnarray}
In the standard measurement error model, the $e_i$ are regarded as IID random variables (this assumption is only reasonable if (i) the measurements are performed in a logically independent manner and (ii) you try hard to ensure that each observation is performed in precisely the same way) with mean 0 (any mean other than 0 gets absorbed into the bias term) and finite standard deviation $\sigma$. Define $\bar{ Y }_n = \frac{ 1 }{ n } \sum_{ i = 1 }^n Y_i$ and $\bar{ e }_n = \frac{ 1 }{ n } \sum_{ i = 1 }^n e_i$.

\begin{itemize}

\item[(a)]

Show that $\bar{ Y }_n = \theta + b + \bar{ e }_n$; show that $V \! \left( \bar{ e }_n \right) = \frac{ \sigma^2 }{ n }$; and therefore show that $E \! \left( \bar{ Y }_n \right) = \theta + b$ and $V \! \left( \bar{ Y }_n \right) = \frac{ \sigma^2 }{ n }$. Intuitively, why is the variance of $\bar{ e }_n$ smaller than the variance of any of the $e_i$ going into $\bar{ e }_n$? Show that your results in this part of the problem imply that $\bar{ Y }_n$ only converges in probability to the truth $\theta$ if $b = 0$. Show that the typical amount RMSE$\left( \bar{ Y }_n \right) = \sqrt{ E \left[ \left( \bar{ Y }_n - \theta \right)^2 \right] }$ by which $\bar{ Y }_n$ is likely to differ from $\theta$ (RMSE stands for \textit{root mean squared error}) is given by the Pythagorean expression
\begin{equation} \label{ph-3}
\text{RMSE} \left( \bar{ Y }_n \right) = \sqrt{ \frac{ \sigma^2 }{ n } + b^2 } \, ,
\end{equation}
and that therefore this also only goes to 0 as more data accumulates if $b = 0$. \textit{[80 points]}

\end{itemize}

\textbf{SOLUTION: }



$Y_i = \theta + b + e_i$
\newline $\theta$ and b are unobservable 

\begin{eqnarray} 
e_i \, \sim  IID N(0, \sigma^2)\nonumber \\
E(Y_i) & = & E(\theta + b + e_i)  \nonumber \\
& = & \theta + b + E(e_i)    \nonumber \\
& = & \theta + b + 0   \nonumber \\
& = & \theta + b  \nonumber \\ \nonumber
\end{eqnarray}
\begin{eqnarray}
V(Y_i) & = & V(\theta + b + e_i) = V(e_i) = r^2 \nonumber \\
Y_i & = & \theta + b + e_i  \nonumber \\
Y_n & = & \frac{1}{n} \sum_{i=1}^{n} Y_i \nonumber \\
& = & \frac{1}{n} \sum_{i=1}^{n} (\theta + b + e_i) \nonumber  \\
& = & (\frac{1}{n} \sum_{i=1}^{n} (\theta)) + (\frac{1}{n} \sum_{i=1}^{n} (b)) + (\frac{1}{n} \sum_{i=1}^{n} (e_i))  \nonumber  \\
& = & \frac{n \theta}{n} + \frac{nb}{n} + (\frac{1}{n} \sum_{i=1}^{n} (e_i)) \nonumber  \\
& = & \theta + b + e_n  \nonumber  \\  \nonumber 
\end{eqnarray}
\begin{eqnarray}
V(\bar e_n) & = & V( \frac{1}{n} \sum_{i=1}^{n} (e_i)) \nonumber  \\
& = & \frac{1}{n^2} V(\sum_{i=1}^{n} (e_i)) \nonumber  \\
& = & IID \, \frac{1}{n^2} \sum_{i=1}^{n} V(e_i) \nonumber  \\
& = & \frac{1}{n^2} \sum_{i=1}^{n} \sigma^2  \nonumber \\
& = & \frac{n \sigma^2}{n^2} = \frac{\sigma^2}{n} \nonumber  \\  \nonumber 
\end{eqnarray}
\begin{eqnarray} 
V(\bar e_n) & = & \frac{\sigma^2}{n} \nonumber \\
E(\bar Y_n) & = & E(\theta + b + \bar e_n)  \nonumber \\
& = & \theta + b + E(\bar e_n) = \theta + b \nonumber  \\
E(e_n) & = & E(\frac{1}{n} \sum_{i=1}^{n} (e_i)) \nonumber  \\
& = & \frac{1}{n} E(\sum_{i=1}^{n} (e_i))  \nonumber \\
& = & \frac{1}{n} E(\sum_{i=1}^{n} 0) = 0  \nonumber \\
V(\bar Y_n) & = & V(\theta + b + \bar e_n)  \nonumber \\
V(\bar e_n) & = & \frac{\sigma^2}{n}  \nonumber \\ \nonumber 
\end{eqnarray}
$\bar Y_n$ has expected value $(\theta + b)$ and variance $\frac{\sigma^2}{n}$\\
$Y_i \sim IID$ mean $(\theta + b)$, variance $\sigma^2 < \infty$ \\
$Y_n \sim IID$ mean $(\theta + b)$, variance $\frac{\sigma^2}{n}$ \\


Weak Law of Large Numbers: $\bar Y_n \xrightarrow[\text{ }]{\text{P}} \theta + b \neq \theta$ unless b=0 meaning the measuring is unbiased, which is not possible and not guaranteed. \\
\begin{eqnarray}
RMSE \left( \bar{ Y }_n \right) & = & \sqrt{ E(\bar Y_n - \theta)^2 }  \nonumber \\
MSE(\bar Y_n) & = & E[(Y_n-\theta)]^2 \nonumber \\
& = & E(\bar Y_n^2 - 2\theta \bar Y-n + \theta^2) \nonumber \\
& = & E(\bar Y_n^2) + E(-2\theta \bar Y-n) + E(\theta^2) \nonumber \\
& = & E(\bar Y_n^2) -2\theta E( \bar Y-n) + \theta^2 \nonumber \\
& = & E(\bar Y_n^2) -2\theta ( \theta +b) + \theta^2 \nonumber \\
 from earlier:  \nonumber \\
V(\bar Y_n) & = & E(\bar Y_n^2) - [E(\bar Y_n)]^2 \nonumber \\
 So V(\bar Y_n) & = & V(\bar Y_n) - [E(\bar Y_n)]^2 \nonumber \\
& = & \frac{\theta^2}{n} - [\theta + b]^2 \nonumber \\
MSE(\bar Y_n) & = & \frac{\sigma^2}{n}+(\theta+b)^2 -2\theta^2 -2\theta + \theta^2 \nonumber \\
& = & \frac{\sigma^2}{n} + b^2 \nonumber \\
MSE(\bar Y_n) & = & \frac{\sigma^2}{n} + b^2 \nonumber \\
RMSE \left( \bar{ Y }_n \right) & = & \sqrt{ E(\bar Y_n - \theta)^2 } \nonumber \\ \nonumber 
\end{eqnarray}
This is similar to Pythagoras's Expression and with more data, the smaller b will be. 



\medskip

Suppose for the rest of this problem that the true pH of this lake is 5.1, so that in fact it's not actually threatened.

\begin{itemize}

\item[(b)]

If you take only a single water sample and process it with your pH kit,
what's the probability that you'll incorrectly conclude that this lake is
threatened? Show your work. \textit{[10 points]}
\medskip
SOLUTION: \\
$SD(e_i) = \sigma = 0.15$\\
$b = 0$\\
$Y_i = 5.1 + 0 + e_i$\\
$e_i \, IID \, N(o, \sigma^2)$\\
\includegraphics[scale=0.7]{F1.JPG} PDF of $Y_i = 5.1 + 0 + e_i$ \\
\huge $\frac{y - \mu}{\sigma} = \frac{5.0-5.1}{0.15} = \frac{-0.1}{0.15} = -0.67$ \\
$\Phi(-0.67)$

\item[(c)]

\normalsize You're not happy with the misclassification probability in (b), and you
decide to remedy this by taking $n > 1$ independent water samples from the
lake and basing your assessment on their mean pH value $\bar{ Y }_n$. How
large does $n$ need to be to make the probability of \{incorrectly
concluding that this lake is threatened\} 0.5\% or less? Be explicit
about all aspects of your probability model, including all of the assumptions you make and whether you think they're reasonable. \textit{[40 points]} 

SOLUTION:\\
From WolframAlpha, P(we say threatened based on $\bar Y_n$, n = 1, truth = 5.0(not threatened)) = $25\%$ (the misclassification error) \\
By making n(number of trials) bigger, the misclassification error rate will go down. \\
\includegraphics[scale=0.5]{F2.JPG} Looking for .0050, PDF of $\bar Y_n$, $n>1$ \\
\huge $\frac{y - \mu}{\frac{\sigma}{\sqrt{2}}} = \frac{5.0-5.1}{\frac{0.15}{\sqrt{n}}}$\\
\normalsize From WolframAlpha, I got $-2.575$ \\
\huge $-2.575 = \frac{5.0-5.1}{\frac{0.15}{\sqrt{n}}}$\\
$n = 14.91$


\end{itemize}

2.~\textit{[140 total points]} (medicine) Hypertension is a medical
condition in which a person's blood pressure is chronically elevated. (A reminder: blood pressure is measured with two numbers called \textit{systolic} (higher) and \textit{diastolic} (lower), in a deeply anachronistic scale called mmHg (millimeters of mercury); blood pressures are stored as data in the form ``systolic over diastolic'' [i.e., 115 over 75 or 115/75; and ideal blood pressures range from 90/60 to 120/80.) Persistent
hypertension is one of the risk factors for strokes, heart attacks, heart
failure and arterial aneurysm, and is a leading cause of chronic renal
failure; as of 1999, it was estimated that 29\% of American adults were
hypertensive. A U.S.~public health goal in 2000 was to lower this rate
to 16\% by 2010, but thing have actually gotten worse since then: the \textit{American Heart Association} estimated in 2018 that \textit{46\%} of all U.S.~adults are hypertensive (although part of the increase is due to a change in the definition of high blood pressure from (above 140 systolic) to (above 130 systolic)). Diet and exercise can go
a long way to lower blood pressure, but drugs are also sometimes needed
(particularly given how hard it is to get Americans to exercise and eat in
a healthier way :-) .

The online reference \textit{Wikipedia} notes that ``\textit{Captopril} is
an angiotensin-converting enzyme (ACE) inhibitor used for the treatment of
hypertension and some types of congestive heart failure. Captopril was the
first ACE inhibitor developed and was considered a breakthrough both
because of its novel mechanism of action and also because of the
revolutionary development process.~...~The development of Captopril was
among the earliest successes of the revolutionary concept of
\textit{structure-based drug design}. The renin-angiotensin-aldosterone
system (a hormone system that helps regulate long-term blood pressure and
blood volume in the body) had been extensively studied in the mid-20th
century, and it had been decided that this system presented several
opportune targets in the development of novel treatments for
hypertension." 

Captopril was developed in the mid 1970s; MacGregor et al.~(1979,
\textit{British Medical Journal}) published the results of a clinical trial
on its effects. Systolic blood pressures (in mmHg) were measured for $n =
12$ representatively-chosen hypertensive patients, before and after taking
Captopril for a long enough time period for the drug to work. Before any data had been gathered, let $( B_i, A_i )$ be a pair of random variables signifying the before and after blood pressure readings for person $i$ in the study (as $i$ runs from 1 to $n$), and define $D_i = ( B_i - A_i )$ and $\bar{ D }_n = \frac{ 1 }{ n } \sum_{ i = 1 }^n D_i$; the realized values of these random variables are given in Table 1.

\begin{table}[t!]

\small

\begin{center}

\begin{tabular}{c|cccccccccccc|cc}

Subject & 1 & 2 & 3 & 4 & 5 & 6 & 7 & 8 & 9 & 10 & 11 & 12 & Mean & SD \\

\hline

Before & 200 & 174 & 198 & 170 & 179 & 182 & 193 & 209 & 185 & 155 & 169
& 210 & 185.3 & 17.1 \\

After & 191 & 170 & 177 & 167 & 159 & 151 & 176 & 183 & 159 & 145 & 146 &
177 & 166.8 & 14.9 \\

\hline

Difference & +9 & +4 & +21 & +3 & +20 & +31 & +17 & +26 & +26 & +10 & +23
& +33 & \ 18.6 & 10.1

\end{tabular}

\caption{\textit{Before and after results for $n = 12$ hypertensive patients
treated with Captopril.}}

\end{center}

\normalsize

\vspace*{-0.35in}

\end{table}

\begin{itemize}

\item[(a)]

Estimate the average effect $\Delta$ of Captopril in the population to
which you believe it's appropriate to generalize here, and explicitly
identify that population. Is this estimated effect large in clinical
terms? Attach a standard error to your estimated effect, and construct an
approximate 95\% confidence interval for $\Delta$, explicitly identifying all assumptions you're making. Is the estimated effect
statistically significant? What do you conclude about Captopril's
usefulness in treating hypertension? Explain briefly. \textit{[80 points]}

SOLUTION:\\
\includegraphics[scale=0.5]{2a1.JPG} \\
At 0, there is no effect on the average of Captopril on the population. \\
Since 0 is not in the 95\% CI, the difference between $\hat{\delta} = 18.6$, and the no-average-effect of 0 is statistically significant implies it's hard to attribute to unlucky random sampling implies that it's possibly real. \\
Conclusion? Captopril works. The catch? This could be a false positive. \\
To decrease the chances of a false positive, we raise the confidence level from 95\% to 99.9\%. (Paraphrased from Professor Draper)\\
\includegraphics[scale=0.35]{2a2.JPG} \\
\includegraphics[scale=0.35]{2a3.JPG} \\
As n goes up, the t curves becomes more like the normal curve. \\
\includegraphics[scale=0.35]{2a4.JPG} $\frac{4.437}{3.2905}$ \includegraphics[scale=0.35]{2a5.JPG} \\
Captopril perhaps does help in treating hypertension. 
\begin{figure}[!b]
\begin{center}

\includegraphics[scale=0.5]{stat-131-take-home-test-3-figure-1.pdf}

\caption{\textit{Scatterplot matrix for the variables \texttt{before},
\texttt{after}, and \texttt{diff}.}}

\vspace*{-0.3in}

\end{center}

\end{figure}

\item[(b)]

Figure 1 presents the scatterplot matrix for the before and after
systolic blood pressure readings on these patients and the differences,
with pairwise correlations noted. 

\begin{itemize}

\item[(i)]

The experimental setup used by the investigators in this problem is called a \textit{repeated-measures} design leading to a \textit{paired comparison}, because blood pressure was measured twice on the same $n$ people and the analysis focused on the differences (before -- after). Another way the experiment could have been run --- this is called a \textit{completely randomized} design --- would be to (I) choose $2 \, n$ hypertensive people in a representative manner and (II) randomize $n$ of them to receive a placebo (the control group) and the other $n$ to receive Captopril (the treatment group). The realized $( B_i, A_i )$ values in Table 1 can be used to make a good guess at what the data set would have looked like if the investigators had used a completely randomized design instead of their paired comparison: the only difference would be that the $B_i$ and $A_i$ values in Table 1 would have been independent, because the data values in column $i$ of the table would have come from two different people. 

The estimate of the treatment effect with the completely randomized design would have been $\hat{ \Delta } = \bar{ B }_n - \bar{ A }_n$, where $\bar{ B }_n = \frac{ 1 }{ n } \sum_{ i = 1 }^n B_i$ and $\bar{ A }_n = \frac{ 1 }{ n } \sum_{ i = 1 }^n A_i$, but notice that this is the same as $\bar{ D }_n = \frac{ 1 }{ n } \sum_{ i = 1 }^n ( B_i - A_i ) = \left( \frac{ 1 }{ n } \sum_{ i = 1 }^n B_i \right) - \left( \frac{ 1 }{ n } \sum_{ i = 1 }^n A_i \right)$. Let $V_{ RM }$ \vspace*{0.05in} and $V_{ CR }$ denote the variance of $\bar{ D }_n$ under the repeated-measures and completely-randomized designs, respectively; also let $\sigma_B^2$ and $\sigma_A^2$ denote the population variances of $B_i$ and $A_i$, respectively, and define $\rho \triangleq \rho ( B_i, A_i )$. Show that
\begin{equation} \label{bp-1}
V_{ RM } \left( \bar{ D }_n \right) = \frac{ \sigma_A^2 + \sigma_B^2 - 2 \, \rho \, \sigma_A \, \sigma_B }{ n } \ \ \ \text{and} \ \ \ V_{ CR } \left( \bar{ D }_n \right) = \frac{ \sigma_A^2 + \sigma_B^2 }{ n } \, ,
\end{equation} 
and that therefore the \textit{efficiency} of the RM design when compared with CR is given by
\begin{equation} \label{bp-2}
e ( RM, CR ) \triangleq \frac{ V_{ CR } \left( \bar{ D }_n \right) }{ V_{ RM } \left( \bar{ D }_n \right) } = \frac{ \sigma_A^2 + \sigma_B^2 }{ \sigma_A^2 + \sigma_B^2 - 2 \, \rho \, \sigma_A \, \sigma_B } \, .
\end{equation} 
Show, using the data values in Table 1 and the correlations in Figure 1, that in this experiment RM was 5.0 times more efficient than CR (and that doesn't even reflect the fact that CR used $2 \, n$ patients instead of the $n$ patients in RM). \textit{[40 points]} \\

SOLUTION: \\
\begin{eqnarray} 
V_{CR} (\bar D_n) & = & V_{CR}(\bar B_n - \bar A_n) \nonumber \\
& \stackrel{I}{=} & V_{CR} (\bar B_n) + (-1)^2 V_{CR}(\bar A_n) \nonumber \\ \nonumber
\end{eqnarray}
$$
B_i \sim IID = \left\{
        \begin{array}{ll}
           E(B_i) = \mu \\
      	   V(B_i) = \sigma_B^2 
        \end{array}
    \right.
$$

$$
A_i \sim IID = \left\{
        \begin{array}{ll}
           E(A_i) = \mu \\
      	   V(A_i) = \sigma_A^2 
        \end{array}
    \right.
$$

\begin{eqnarray} 
V_{CR}(\bar B_n) & = & \frac{\sigma_B^2}{n} \nonumber\\
\bar A_n & = & \frac{1}{n} \sum_{i=1}^{n} A_i \nonumber \\
V_{CR} & = & \frac{\sigma^2_A}{n} \nonumber \\ \nonumber
\end{eqnarray}

So, $V_CR (\bar D_n) = \frac{\sigma^2_A + \sigma^2_B}{n}$

\begin{eqnarray} 
V_{RM} & = & V_{RM} (\frac{1}{n} \sum_{i=1}^{n} [B_i - A_i]) \nonumber \\
& = & \frac{1}{n^2} V_{RM} [\sum_{i=1}^{n} [B_i - A_i]] \nonumber \\ 
& = & \frac{1}{n^2} [\sum_{i=1}^{n} V_{RM} [B_i - A_i]] \nonumber \\
\nonumber \\
V_{RM} (B_i - A_i) & = & V_{RM}(B_i) + V_{RM}(A_i) + 2 C_{RM} (B_i , -A_i)\nonumber \\
& = & V_{RM}(B_i) + (-1)^2 V_{RM}(A_i) - 2 C_{RM} (B_i , -A_i) \nonumber \\ \nonumber
\end{eqnarray}

\begin{eqnarray} 
\frac{C_{RM}(B_i , -A_i)}{SD(A_i) * SD(B_i)} = \rho_{AB} \nonumber \\ \nonumber
\end{eqnarray}

\begin{eqnarray} 
V_{RM} & = & \frac{1}{n^2} \sum_{i=1}^{n} V_{RM}(B_i - A_i)  \\
& = & \frac{1}{n^2} \sum_{i=1}^{n} [\frac{\sigma^2_A}{V(A_i)} + \frac{\sigma^2_B}{V(B_i)}] - 2 \rho_{AB} \sigma_A  \sigma_B \nonumber \\
& = & \frac{1}{n^2} \sum_{i=1}^{n} ( \sigma^2_A + \sigma^2_B - 2 \rho_{AB}  \sigma_A  \sigma_B) \nonumber \\
& = & \frac{\sigma^2_A + \sigma^2_B - 2 \rho_{AB}  \sigma_A  \sigma_B}{n} \nonumber \\ \nonumber 
\end{eqnarray}

$2 \rho_{AB} = 0.8084 $




\item[(ii)]

Does the effect of the drug seem to be constant across the 12 patients, or
is there a tendency for the drug to have a larger or smaller effect for
people whose initial blood pressure was high than for those whose initial
reading was lower? Which (if any) of the correlations in Figure 1 supports this
conclusion? Explain briefly. \textit{[20 points]}

\end{itemize}

SOLUTION: \\
The drug doesn't seem constant with the 12 patients as there is a tendency for the higher the patient's before number, the better the improvement. Everyone improved under Captopril but the difference numbers range from +3 to +33. This is a really big difference as everyone had different improvements. Patient 4's before number is unusual as it's the 3rd smallest before number and had a small improvement. While patient 12 has a large before number and a large improvement. It's not always the case but that

\end{itemize}

\end{document}